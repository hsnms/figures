\documentclass[nottitlepage]{article}
%\usepackage{amsmath,amssymb,bm,epsfig,graphics,verbatim}
\usepackage{geometry}
\geometry
{
letterpaper,
top=1in,
bottom=1in,
left=1in,
right=1in,
}
\linespread{1.3}

\begin{document}
\title{Personal Statement}
\author{Shengnan Huang\\
\\
\multicolumn{1}{p{.8\textwidth}}{\centering\emph{School of Physics, Georgia Institute of Technology,
\\837~State Street, Atlanta, GA~30332}}}
\date{}
\maketitle
One year after starting my PhD in physics, I decided to take some graduate math courses. On one hand, I'd like to choose challenging courses to satisfy the minor course requirements. On the other hand, I thought these courses could help me become a good physicist, as I witnessed the mathematical prowess of my advisor, Prof. Paul Goldbart, who shocked me with his rich mathematical knowledge. I was intrigued by the term ``measure" he often mentioned when writing integrals. At that time, I already considered it late to master these advanced mathematical concepts. But I convinced myself that I was pursuing a long academic career, and I should have the courage to try, leaving no regrets.

As it turned out, I was quite fascinated by the exotic style of advanced mathematics. I have begun to appreciate the bizarre-looking formulas as well as the infinite variety of techniques used in different proofs, which I didn't expect to see that many in physics.  Perhaps, this is due to the fact that I always have an appetite for style and techniques in literature: when I am serious about reading a novel, a great story is not my priority; instead, I am looking for a style enhanced by artistic techniques that set it apart. Here, I have found an analog of my long-time hobby. 

I came to realize that math can be fun and creative. The mathematical reasoning not only requires me to think logically, but also encourages me to think creatively and imaginatively. Many times, when I was doing homework during weekends, I initially did not know how to start the proof. If this happened in the past, I may have been panic. But here I tried to calm down and keep focused. Then after pondering and pondering, I experienced the magical moments when I suddenly could see the structure of the proof laying out. I had similar experience when I was a primary school kid, attracted by some music while afraid of not getting the money to buy cassettes, tried to use the pipe learned from school to capture the music score. There were those similar magical times when I suddenly captured the first few notes and harvested the whole song soon after. Here, I can enjoy the moments of creating a piece of logical work.

This experience helps to shape my problem-solving style, which is intuition-oriented. I like to brainstorm a wide range of perspectives and then think intuitively about a structure or a story of the solution before filling it up with details. I also like to have a vague picture of the new techniques before doing literature search. I believe this is an active participation in the thought process of a problem, which can help me discover the essence of the problem really quickly and make the solution look more natural. %One time, Paul told me that I had a tendency of not using the principle. That was because I was still dwelling at step 2. I'm not interested in reformulating a theory or trying to find an example to use the theory.

This style has brought me with some good progress in research. For one of my first projects, I was trying to derive the propagator in the vulcanization theory by using Dyson equation, which required me to compute many terms perturbatively. After months of intense calculations, I braked my mind and took a holistic view of the problem, which helped me later to figure out an iterative method that was non-perturbative in nature. 

However, in later research, I had difficulty in using my problem-solving style as well as balancing the research styles of mathematics and physics. For the study of defects in nematic liquid-crystal capillary bridges, I wanted to do more besides numerical computations, in order to feel the excitements of creation. This motivated me to design an analytical theory. The initial picture came to my mind was that the defect cores resembled holes in a manifold, therefore the problem was about the prediction of the locations of these holes. Inspired by ideas of quantum field theory, I was looking for a formalism where the defect cores can be created or annihilated. But later, I was frustrated by the fact that I underestimated the difficulty of solving those nonlinear PDEs, as well as the fact that many analytical theories are based on special simplifications such that analytical solutions can be obtained. After lots of thoughts, I started to question whether a thorough solution of the PDEs was necessary. However, it took me a very long time to figure out that the Fourier series method was promising and start to experiment with it on simple one-dimensional examples. 

A lesson I took from this experience is about how to deal with the weakness of my problem-solving style. When good inspirations are sporadic and imaginations become irrelevant, I need to keep feet on the ground and gather concrete evidence through detailed calculations. As in this experience, initially I thought one-dimensional examples were trivial and boring. But soon I found out that the calculations were unexpectedly challenging and fun. As calculations further developed, I can see patterns in my initial picture starting to form: locations of discontinuities in one-dimensional examples are related to the dominant Fourier modes of the background energy; correspondingly, the number and locations of nematic defects may be determined by only a finite number of Fourier modes, and these modes may be created or annihilated.  

A question from this experience is about which kind of research I'm truly passionate about. I realize that my joy is not from the results I obtain; rather, it is the way leading to these results that excites me. I chose physics because making predictions was exciting. However, compared with Einstein's prediction of gravitational waves, I'm more interested in the fact that Riemann's differential geometry can provide the language for general relativity. Moreover, even though I always want to express uniqueness in my research, I may not have to expect the incidents that my initial mind pictures appearing drastically different from the literature in my research field.

Now, many years after taking my first graduate math course, I decided to apply for a math PhD. I particularly like topology and geometry. On one hand, Algebraic Topology and Algebraic Geometry were the two most challenging and exciting courses I took; and I had a sense of flying during the process of learning, which I expect more for my future research. On the other hand, as a physics student trained for many years, I like to participate in the math research where my physical insights may help; and topology $\&$ geometry is the field where both mathematicians and physicists are excited about.

As a physics student trained for many years and the one who likes mathematics, I have curiosity of quantum field theory. Sometimes I'm wondering if there is a more fundamental theory underlying the Feynman's formalism, which can explain why there is a wave-particle duality and why there are bosons and fermions. This ambition may be too big, as I can remember when I first came to Georgia Tech for PhD in physics, I told my advisor that I was impressed by Bohm's interpretation of quantum mechanics. He said first I may not solve it and second even if I can solve it people may not understand it. 



\end{document}